%%%%%%%%%%%%%%%%%%%%%%%%%%%%%%%%%%%%%%%%%
% Short Sectioned Assignment
% LaTeX Template
% Version 1.0 (5/5/12)
%
% This template has been downloaded from:
% http://www.LaTeXTemplates.com
%
% Original author:
% Frits Wenneker (http://www.howtotex.com)
%
% License:
% CC BY-NC-SA 3.0 (http://creativecommons.org/licenses/by-nc-sa/3.0/)
%
%%%%%%%%%%%%%%%%%%%%%%%%%%%%%%%%%%%%%%%%%

%----------------------------------------------------------------------------------------
%	PACKAGES AND OTHER DOCUMENT CONFIGURATIONS
%----------------------------------------------------------------------------------------

\documentclass[paper=a4, fontsize=11pt]{scrartcl} % A4 paper and 11pt font size

\usepackage[T1]{fontenc} % Use 8-bit encoding that has 256 glyphs
\usepackage{fourier} % Use the Adobe Utopia font for the document - comment this line to return to the LaTeX default
\usepackage[english]{babel} % English language/hyphenation
\usepackage{amsmath,amsfonts,amsthm,mathtools,amssymb} % Math packages
\usepackage[margin=1.75cm]{geometry}
\usepackage{multicol}
\usepackage{setspace}
\usepackage{graphicx}
\usepackage{setspace}
\onehalfspacing
\usepackage{multicol}
\allowdisplaybreaks
\usepackage{hyperref}

% Binomial tree
\usepackage{tikz}
\usetikzlibrary{matrix}

% Permutations and combinations 
\newcommand*{\Perm}[2]{{}^{#1}\!P_{#2}}%
\newcommand*{\Comb}[2]{{}^{#1}C_{#2}}%

\usepackage{sectsty} % Allows customizing section commands
\allsectionsfont{\raggedright \normalfont\scshape} % Make all sections left, the default font and small caps
\renewcommand{\thesubsection}{\alph{subsection}} % Make the subsections letters
\newcommand{\rreduce}[2]{\mathop{\longrightarrow}_{\tiny{#1}}}

\usepackage{fancyhdr} % Custom headers and footers
\pagestyle{fancyplain} % Makes all pages in the document conform to the custom headers and footers
\fancyhead{} % No page header - if you want one, create it in the same way as the footers below
\fancyfoot[L]{} % Empty left footer
\fancyfoot[C]{} % Empty center footer
\fancyfoot[R]{\thepage} % Page numbering for right footer
\renewcommand{\headrulewidth}{0pt} % Remove header underlines
\renewcommand{\footrulewidth}{0pt} % Remove footer underlines
\setlength{\headheight}{13.6pt} % Customize the height of the header

\numberwithin{equation}{section} % Number equations within sections (i.e. 1.1, 1.2, 2.1, 2.2 instead of 1, 2, 3, 4)
\numberwithin{figure}{section} % Number figures within sections (i.e. 1.1, 1.2, 2.1, 2.2 instead of 1, 2, 3, 4)
\numberwithin{table}{section} % Number tables within sections (i.e. 1.1, 1.2, 2.1, 2.2 instead of 1, 2, 3, 4)

\setlength\parindent{0pt} % Removes all indentation from paragraphs - comment this line for an assignment with lots of text

%----------------------------------------------------------------------------------------
%	TITLE SECTION
%----------------------------------------------------------------------------------------

\newcommand{\horrule}[1]{\rule{\linewidth}{#1}} % Create horizontal rule command with 1 argument of height

\title{	
\normalfont \normalsize 
\textsc{University of Chicago | Financial Mathematics} \\ [25pt] % Your university, school and/or department name(s)
\horrule{0.5pt} \\[0.4cm] % Thin top horizontal rule
\huge Vasicek \\ % The assignment title
\horrule{2pt} \\[0.5cm] % Thick bottom horizontal rule
}

\author{Michael Beven - 455613} % Your name

\date{\normalsize\today} % Today's date or a custom date

\begin{document}

\maketitle % Print the title

%----------------------------------------------------------------------------------------
%	PROBLEM 1
%----------------------------------------------------------------------------------------

\break

The Vasicek model is given by:

\begin {align*}
dr(t) = \alpha(\mu - r(t))\mathrm{d}t + \sigma \mathrm{d} \tilde W(t)
\end{align*}

Where $r(t)$ is the risk-free rate of interest.  For $s > 0$, $r(t+s)$ given $r(t)$ is Normally distributed under the physical probability measure with mean $\mu + (r(t) - \mu)e^{- \alpha s}$ and variance $\frac{\sigma^2 (1 - e^{-2\alpha s})}{2\alpha}$.

Prices for zero-coupon bonds are given by:

\begin{align*}
P(t,T) = e^{A(t,T) - B(t,T)r(t)}
\end{align*}

where

\begin{align*}
B(t,T) &= \frac{1-e^{-\alpha (T - t)}}{\alpha}\\
A(t,T) &= (B(t,T) - (T - t))\Bigg( \mu - \frac{\sigma^2}{2\alpha^2}\Bigg) - \frac{\sigma^2}{4\alpha}B(t,T)^2
\end{align*}

The price of the European call which matures at time $S$ with strike $K$ and exercise date $T$ (with $T < S$) is:

\begin{align*}
V_c(t) &= P(t,S)\Phi (d_1) - K P(t,T) \Phi(d_2)
\end{align*}

where

\begin{align*}
d_1 &= \frac{1}{\sigma_p} log \frac{P(t,S)}{KP(t,T)} + \frac{\sigma_p}{2}, \hspace{10mm} d_2 = d_1 - \sigma_p\\
\sigma_p &= \frac{\sigma}{\alpha} (1 - e^{-\alpha (S - T)}) \sqrt{\frac{1 - e^{-2\alpha(T-t)}}{2\alpha}}
\end{align*}

By put-call parity, we also have the price of the corresponding European put as:

\begin{align*}
V_p(t) &= K P(t,T) \Phi(-d_2) - P(t,S)\Phi (-d_1)
\end{align*}

%----------------------------------------------------------------------------------------

\end{document}